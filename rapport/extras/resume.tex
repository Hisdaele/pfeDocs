\chapter*{Résumé}
L’émergence des grandes villes au Maroc vient avec ses lots de difficultés parmi lesquelles les problèmes de mobilité dus à l’augmentation fulgurante du nombre de véhicules  en circulation. Pour faire face à quelques unes de ces difficultés qui influencent le trafic routier, les systèmes de reconnaissance automatique des plaques d’immatriculation ou systèmes ANPR peuvent être déployés. A partir des images prises par une caméra, ces systèmes peuvent détecter automatiquement les matricules des plaques pour les enregistrer sous format de données structurées (textes enregistrables dans une base de données). Notre projet de fin d’études effectué au sein de la jeune entreprise marocaine KF2Y Consulting visait une modernisation des villes marocaines grâce au développement d’un système ANPR propre aux plaques d’immatriculation typiquement marocain. Ce système comprend deux composants: un composant pour la localisation des plaques d’immatriculation sur les images et un composant pour la lecture des caractères sur les images des plaques détectées. Ces deux composants ont été créés grâce à l’entraînement d’un modèle Deep Learning de détection des objets en utilisant l’algorithme YOLOv4. Si le modèle pour la localisation de plaques a atteint une précision de plus de 99\% , le modèle de lecture des caractères a une précision de plus de 96\%; ce qui donne au système une précision de plus de 95\%. Par la suite, le système ANPR que nous avons appelé MoPlaZer a été intégré dans une application mobile que nous avons développée. C’est une application sous Android qui permet de faire la reconnaissance sur les images capturées par l’utilisateur ou des vidéos en temps réel. En outre, le système MoPlaZer a été déployé dans un système embarqué pour parking intelligent. Dans ce parking, MoPlaZer sert à ouvrir automatiquement la barrière donnant accès aux espaces de stationnement à partir de la lecture du numéro d’immatriculation d’un véhicule préalablement enregistrée à partir d’une plateforme mobile de réservation de places dans les parkings.


\textbf{Mots clés}: villes intelligentes, mobilité intelligente, système ANPR, Machine Learning, Détection des objets, YOLO, Android, parking intelligent, système embarqué.


\chapter*{Abstract}
The emergence of large cities in Morocco comes with its share of difficulties, including mobility problems due to the increase in the number of vehicles in circulation. To deal with some of these difficulties that influence road traffic, automatic number plate recognition systems or ANPR systems can be deployed. From the images taken by a camera, these systems can automatically detect the license plates in order to save them in structured data format (texts that can be recorded in a database). Our end of studies project carried out within the young Moroccan company KF2Y Consulting aimed at modernizing Moroccan cities through the development of an ANPR system specific to typical Moroccan number plates. This system consists of two components: a component for localizing license plates on images and a component for reading characters on images of detected plates. These two components were created through training an object detection deep learning model using the YOLOv4 algorithm. If the model for the location of plaques has reached an accuracy of more than 99\%, the character reading model has an accuracy of more than 96\%; which gives the system an accuracy of over 95\%. Subsequently, the ANPR system that we called MoPlaZer was integrated into a mobile application that we developed. It is an Android application that allows recognition on user-captured images or videos in real time. In addition, the MoPlaZer system has been deployed in an embedded system for smart parking. In this car park, MoPlaZer is used to automatically open the barrier giving access to parking spaces by reading the registration number of a vehicle previously registered using a mobile parking space reservation platform.


\textbf{Keywords}: smart cities, smart mobility, ANPR system, Machine Learning, Object detection, YOLO, Android, smart parking, embedded system.