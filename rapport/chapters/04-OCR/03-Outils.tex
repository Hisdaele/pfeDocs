\section{Outils}
De nombreuses \acrshort{api} et bibliothèques ont été développées pour effectuer les opérations d’OCR sur les documents. Parmi elles, nous pouvons citer:

\begin{itemize}
    \item \textbf{Tesseract}:c’est un moteur de reconnaissance optique de caractères écrit en C et C++ pour divers systèmes d'exploitation (Linux, Windows et Mac OS X). Il s'agit d'un logiciel gratuit, publié sous la licence Apache, et le développement est parrainé par Google depuis 2006. En 2006, Tesseract était considéré comme l'un des moteurs OCR open source les plus précis. Avec la version 4, un moteur et des modèles OCR basés sur LSTM (Long-Short Term Memory) pour de nombreuses langues et scripts supplémentaires ont été ajoutés, portant ainsi le nombre de langues traitées à 116.
    \item \textbf{EasyOCR}: comme Tesseract, EasyOCR est une bibliothèque  open source d’OCR. Développé par JaidedAI , il prend en charge plus de 80 langues et tous les scripts d'écriture courants, notamment le latin, le chinois, l'arabe, le cyrillique, etc. Par rapport à Tesseract, il possède l’avantage d’être rapide lors d’une exécution avec le GPU.
    \item \textbf{MLKit Text Recognition}: 
    C’est API développé par Google pour l’OCR. Elle est principalement utilisée dans les applications mobiles et prend uniquement en compte les caractères latins.
    
\end{itemize}
