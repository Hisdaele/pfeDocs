\addcontentsline{toc}{chapter}{Introduction Générale}
\chapter*{\textbf{Introduction Générale}}
Selon une publication parue le 28 février 2020 au quotidien d'informations Aujourd'hui Le Maroc, le parc automobile marocain comptait en 2018 environ 4,3 millions de véhicules en circulation. Ceci correspond à une augmentation de plus de 50\% par rapport à 2002 où on dénombrait environ 1,81 million de véhicules. Cette croissance rapide et continue n’est pas sans conséquence sur la société. On peut citer notamment la montée des fléaux tels que les vols de voitures, la violation du trafic routier, les collisions, les congestions et sans compter leurs impacts sur l’économie nationale. Face à cette situation, la mise en place des systèmes automatiques et performants de gestion du trafic routier devient cruciale. Un des éléments de haute importance qui entre dans ces systèmes est l’identification des véhicules. Et quoi de plus simple et efficace pour identifier les véhicules que les plaques d’immatriculation. Encore appelée plaque minéralogique, une plaque d’immatriculation est un objet généralement en forme rectangulaire placé sur un véhicule. Sur elle, est inscrite une combinaison unique de chiffres et de lettres qui contient des informations sur un véhicule et par ricochet sur son propriétaire. 

Ces dernières années, les avancées technologiques dans le domaine de l’\acrshort{ia} et l'émergence des citées dites intelligentes ont donné un regain d'intérêt aux chercheurs et entreprises internationaux en général et marocains en particulier sur la question de l’identification des véhicules via leur plaque. Dans le métier, on  parle le plus souvent d’un dispositif de \acrfull{lapi} ou encore en anglais \acrfull{anpr}. C’est une technologie d’identification qui utilise la plupart du temps des techniques de traitements d’images , de vision par ordinateur (Computer Vision en anglais) et de reconnaissance optique de caractères ou \acrfull{ocr} pour lire les plaques d’immatriculation de véhicules. Toutefois, avec l’essor de l’\acrshort{ia} qui fait déjà largement ses preuves dans plusieurs domaines (la médecine, l’industrie, l’aviation), des techniques modernes et plus performantes utilisant les algorithmes puissants d’apprentissage automatique (Machine Learning) se sont ajoutées au processus d’\acrshort{anpr}. 

Dans ce sens, il existe déjà sur le marché plusieurs solutions d’\acrshort{anpr} qui sont soit intégrées dans des caméras adaptées soit "consommables" via les services web. Si d’une part, la plupart de ces solutions sont payantes, d'autre part elles ne traitent généralement pas le cas des modèles de plaques marocaines qui contiennent des caractères en arabe. C’est dans ce cadre que s’inscrit notre projet de fin d’études: mettre en place une solution de reconnaissance automatique de plaques minéralogiques marocaines en intégrant les algorithmes de Machine Learning.

Le présent rapport qui fait étalage des travaux que nous avons réalisés est composé de quatre(4) chapitres. Le premier chapitre sera consacré à la présentation de l’organisme d’accueil, du projet de manière générale ainsi que de la méthodologie de travail suivie. Un panorama synthétique et organisé des travaux déjà réalisés sur les systèmes d’\acrshort{anpr} sous forme d'état de l’art sera fait dans le deuxième chapitre. Le troisième chapitre nous permettra de faire étalage en profondeur de l’approche que nous avons adoptée. La mise en œuvre ou encore l’implémentation de notre solution sera exposée au dernier chapitre. Enfin en guise de conclusion générale, nous donnerons une synthèse des travaux suivie des perspectives.