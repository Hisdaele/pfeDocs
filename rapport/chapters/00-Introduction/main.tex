\addcontentsline{toc}{chapter}{Introduction Générale}
\chapter*{\textbf{Introduction générale}}
Selon une publication parue le 28 février 2020 au quotidien d'informations \textbf{\textit{Aujourd'hui LeMaroc}}, le parc automobile marocain comptait en 2018 environ \textbf{4,3 millions de véhicules en circulation}. Ceci correspond à une augmentation de plus de 50\% par rapport à
l'année 2002 où on dénombrait environ 1,81 million de véhicules \cite{ajLeMaroc}. Cette croissance rapide et continue n’est pas sans conséquence sur la société. On peut citer notamment la montée des fléaux tels que les \textbf{vols de voitures}, la \textbf{violation du trafic routier}, les \textbf{collisions}, les \textbf{congestions} et sans compter leurs impacts sur l’économie nationale. Face à cette situation, la mise en place des systèmes automatiques et performants de gestion du trafic routier devient cruciale. Un des éléments de haute importance qui entre dans ces systèmes est l’\textbf{identification des véhicules}. Et quoi de plus simple et efficace pour identifier les véhicules que les plaques d’immatriculation. Encore appelée plaque minéralogique, une \textbf{plaque d’immatriculation} est un objet généralement en forme rectangulaire placé en avant et/ou en arrière d'un véhicule. Sur elle, est inscrite une combinaison unique de chiffres et/ou de lettres qui contient des informations sur un véhicule et par ricochet sur son propriétaire. 

Ces dernières années, les avancées technologiques dans le domaine de l’\acrfull{ia} et l'émergence des \textbf{villes dites intelligentes} ont donné un regain d'intérêt aux chercheurs et entreprises internationaux en général et marocains en particulier sur la question de l’identification des véhicules à travers leur plaque. Dans le métier, on  parle le plus souvent d’un dispositif de \acrfull{lapi} ou encore en anglais \acrfull{anpr}. C’est une technologie d’identification qui utilise la plupart du temps des techniques de \textbf{traitements d’images} , de \textbf{vision par ordinateur} et de \textbf{reconnaissance optique de caractères} ou \acrfull{ocr} pour lire les matricules inscrites sur les plaques de véhicules. Par ailleurs, avec l’essor de l’IA qui fait déjà largement ses preuves dans plusieurs domaines (la médecine, l’industrie, l’aviation), des techniques modernes et plus performantes utilisant les algorithmes puissants d’\textbf{apprentissage automatique (Machine Learning)} se sont ajoutées au processus d’\acrshort{anpr}. 

En réalité, on retrouve déjà sur le marché plusieurs solutions d’\acrshort{anpr} qui sont soit intégrées dans des caméras adaptées soit "consommables" via les services web. Si d’une part, la plupart de ces solutions sont payantes, d'autre part elles ne traitent généralement pas le cas des modèles de plaques marocaines qui contiennent des caractères en arabe. C’est dans ce cadre que s’inscrit notre projet de fin d’études: \textbf{mettre en place un \acrfull{siram}} et \textbf{le déployer dans des applications bien précises}.

Ce document qui fait étalage des travaux réalisés est découpé en \textbf{7 chapitres}. Le premier chapitre sera consacré à la  \textbf{présentation du contexte général}. Un \textbf{panorama synthétique et organisé des travaux et sujets déjà traités autour des systèmes ANPR} sera fait en 4 chapitres très détaillés et illustratifs. Le sixième chapitre présentera la \textbf{conception et la mise en œuvre de notre solution SIRAM}. Enfin le septième et dernier chapitre montrera les \textbf{diverses applications réalisées de notre solution}.