\section{Phases d'un cycle de Machine Learning}
Pour construire un modèle de Machine Learning qui répond à un problème bien défini au préalable, il est indispensable de suivre une chaîne qui se décline en 5 phases:
    \begin{enumerate}
        \item \textbf{Acquisition des données}: la possession de données historiques est la source de tout apprentissage. De plus, plus la quantité est élevée, meilleur sera l’apprentissage. Ces données peuvent provenir de plusieurs sources différentes: Web, sources internes et bien d'autres encore.
        \item \textbf{Nettoyage, Préparation, Manipulation des données}: les données ont besoin d’une retouche avant d’être utilisées. En effet, certains attributs sont inutiles, d’autre doivent être modifiés afin d’être compris par l’algorithme et enfin, certains éléments sont inutilisables car leurs données sont incomplètes. Plusieurs techniques telles que la Data Visualisation, la Data Transformation ou encore la Normalisation permettent de gérer cette problématique.
        \item \textbf{Création du modèle}: une fois les données clarifiées, un ou plusieurs modèles sont créés à l’aide d’algorithmes de Machine Learning.
        \item \textbf{Évaluation}: une fois le modèle réalisé, il faut l’évaluer pour prouver son efficacité et sa performance. En utilisant un second jeu de données, on identifie la précision du modèle et on répète l’étape 3 et 4 jusqu’à l’obtention de la meilleure performance.
        \item \textbf{Déploiement}: la dernière étape est la mise en production du modèle. Il est désormais possible de l’utiliser pour les nouvelles données entrantes. Le système est itératif puisque ces nouvelles données sont réutilisées par l’algorithme pour améliorer le modèle. Le système est en perpétuel évolution et amélioration.
    \end{enumerate}