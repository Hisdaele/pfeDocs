\section{Méthodes d'apprentissage automatique}
La création d’un modèle se fait à l’aide d’un algorithme de Machine Learning spécifique au type de problème à résoudre.  On classe généralement ces algorithmes en plusieurs classes dont 3 sont les plus connues à l’heure actuelle.

    \subsection{Apprentissage supervisé}
    Les algorithmes d'apprentissage automatique supervisés peuvent appliquer ce qui a été appris dans le passé à de nouvelles données en utilisant des exemples étiquetés pour prédire des événements futurs. À partir de l'analyse d'un ensemble de données d'apprentissage connu, l'algorithme d'apprentissage produit une fonction inférée pour faire des prédictions sur les valeurs de sortie. Le système est capable de fournir des objectifs pour toute nouvelle entrée après une formation suffisante. L'algorithme d'apprentissage peut également comparer sa sortie avec la sortie correcte et prévue et trouver des erreurs afin de modifier le modèle en conséquence.\cite{expertAi}
    Selon le type de variable à prédire, on distingue deux types d’apprentissage supervisé: la \textbf{classification} pour les variables qualitatives, discrètes et la \textbf{régression} pour les variables quantitatives, continues.
    Les algorithmes les plus connus d’apprentissage supervisé sont:
    \begin{itemize}
        \item La \textbf{machine à vecteurs de support} (\acrshort{svm} en anglais)
        \item La \textbf{méthode des k plus proches voisins} (\acrshort{knn} en anglais)
        \item L'\textbf{arbre de décision}
        \item La \textbf{classification naïve bayésienne}
        \item Les \textbf{réseaux de neurones}
    \end{itemize}

    \subsection{Apprentissage non-supervisé}
    Les algorithmes d'apprentissage automatique non supervisés ont utilisés lorsque les informations utilisées pour l'entraînement ne sont ni classées ni étiquetées. L'apprentissage non supervisé étudie comment les systèmes peuvent déduire une fonction pour décrire une structure cachée à partir de données non étiquetées. Le système ne trouve pas la bonne sortie, mais il explore les données et peut tirer des inférences à partir d'ensembles de données pour décrire des structures cachées à partir de données non étiquetées.\cite{expertAi}
    Parmi les algorithmes d'apprentissage non-supervisé, on retrouve:
        \begin{itemize}
            \item \textbf{K-Moyenne} (K-Means en anglais)
            \item La \textbf{réduction de la dimensionnalité}
            \item L'\textbf{analyse des composants principaux}
        \end{itemize}

    \subsection{Apprentissage par renforcement}
    Le renforcement des algorithmes d'apprentissage automatique est une méthode d'apprentissage qui interagit avec son environnement en produisant des actions et découvre des erreurs ou des récompenses. La recherche par essais et erreurs et la récompense différée sont les caractéristiques les plus pertinentes de l'apprentissage par renforcement. Cette méthode permet aux machines et aux agents logiciels de déterminer automatiquement le comportement idéal dans un contexte spécifique afin de maximiser ses performances. Un simple retour de récompense est nécessaire pour que l'agent sache quelle action est la meilleure ; c'est ce qu'on appelle le signal de renforcement.\cite{expertAi}




L’une des applications de ces algorithmes se trouve dans le domaine de la vision par ordinateur avec la reconnaissance objets sur une image. Et pour cela, on utilise général une sous-branche du Machine Learning appelée Deep Learning ou encore apprentissage profond.