\addcontentsline{toc}{chapter}{Conclusion Générale}
\chapter*{Conclusion Générale}
Toutes les grandes villes du monde en général et du Maroc en particulier font face de nos jours à une forte croissance du nombre de véhicules en circulation. Ceci engendre des difficultés majeures sur la fluidité de la mobilité au sein de ces villes. Avec la venue de l’intelligence artificielle, le concept des villes intelligentes a émergé pour pallier plus ou moins complètement à ces difficultés. Encore appelées Smart Cities, ces villes se composent de plusieurs dispositifs utilisant les nouvelles technologies en informatique pour améliorer la qualité de services urbains en les automatisant au maximum. Parmi ces dispositifs on retrouve les systèmes de reconnaissance automatique des plaques d’immatriculation  ou ANPR qui servent à lire les matricules des véhicules sur les images. 


La mission qui nous a été confiée pour notre projet de fin d’études au sein de la dynamique et jeune entreprise marocaine portait essentiellement sur \textbf{le développement d’un système ANPR exclusivement pour les plaques d’immatriculation marocaines}. Durant notre stage qui s’est étalé \textbf{6 mois (15 février au 15 août 2021)}, nous avons eu recours aux nouvelles techniques d’apprentissage automatique et de traitement d’images  pour mettre en place un système ANPR alliant rapidité et précision. Ce système a été déployé d’abord dans une application mobile développée sous Android et par la suite un système embarqué par une maquette de parking intelligent.


Ce document bien plus qu’un simple rapport tient lieu de références pour comprendre l’ensemble des travaux réalisés durant cette période. Période au cours de laquelle nous avons mis le  savoir et savoir-faire acquis au cours d’une longue et enrichissante formation en ingénierie informatique au service des besoins de l’entreprise et par extension de la société. Il contient 7 chapitres qui peuvent être subdivisés en trois grandes parties. La première partie comportant un chapitre et le premier présente le contexte général du projet. La deuxième partie s'étend sur 4 chapitres (les chapitres 2 à 5). Partant de l’étude des systèmes ANPR jusqu’à l’apprentissage automatique en passant le traitement d’images et la reconnaissance optique de caractères, nous avons exploré les travaux réalisés  dans le domaine de notre sujet. La troisième partie comporte les deux derniers chapitres qui exposent la partie pratique de notre projet. Dans le premier, nous avons montré l’architecture de notre système ANPR appelé MoPlaZer qui est composé d'une partie de détection de la plaque d’immatriculation réalisée  et d’une partie de lecture du matricule de la plaque détectée. Toutes ces deux parties ont été implémentées en entraînement des modèles de détection d’objets avec YOLOv4. Le modèle de détection atteint une \textbf{précision de plus 99\%} et le modèle de lecture une \textbf{précision d'environ 96\%}. Le système MoPlaZer a donc une \textbf{précision de 95\%}. Dans le dernier chapitre, nous avons présenté les deux grandes applications dans laquelle nous avons déployé notre système MoPlaZer: une application mobile développée sous Android et un smart parking utilisant une barrière intelligente pour donner accès ou non aux zones de stationnement.


Notre projet de fin d’études au-delà d’un simple stage a été une expérience professionnelle enrichissante mais aussi parsemée de quelques difficultés. Enrichissante dans la mesure où nous avons pris en main pendant 6 mois un projet d’entreprise qui a une grande différence des projets personnels ou scolaires. Nous avons ainsi eu l’occasion d'explorer d'autres domaines hors de notre formation académique à savoir l’intelligence artificielle et les systèmes embarqués. Ceci nous a donc poussé à apprendre des nouveaux outils (TensorFlow, Kotlin, Colab, LabelImg et bien d’autres). Par ailleurs si d’une part ce stage nous a permis de renforcer les compétences acquises à l’école d’autre part nous avons développé de nouveaux aptitudes comme le respect des horaires de travail, le travail en équipe dans une entreprise, la mise en place d’une méthodologie de résolution de problèmes, une bonne et efficace lecture des articles et des travaux des chercheurs. Toutefois nous avons rencontré de nombreux challenges comme l’ignorance de certains outils, la difficulté à mettre en place un programme de travail objectif et aussi le respect des délais des tâches. Mais avec le temps, la persévérance nous avons réussi à relever de plus en plus ces challenges et réussir à satisfaire les responsables en entreprise. 


Notre projet de fin d’études conclut notre formation d’ingénierie mais nos travaux ne sauraient être limités là. D’autres pistes peuvent être explorées pour améliorer les résultats obtenus à la fin de notre stage. Ces pistes qui sont des perspectives peuvent être situées à trois niveaux. Au niveau du système ANPR, nous pouvons utiliser au lieu d’un modèle lourd et lent de détection YOLO pour faire la lecture des caractères sur les plaques, la combinaison de la segmentation des caractères et d’un modèle de classification de caractères. Ceci est conditionné par la nécessité de collecter plus d’images de plaques d’immatriculation marocaines pour l'entraînement du modèle de classification. Au niveau de l’application mobile pour déployer le système MoPlaZer, les perspectives peuvent être: ajout de nouvelles fonctionnalités comme la sauvegarde des matricules détectées, le développement de la version iOS de l’application et l’optimisation des traitements en parallèle pour réduire le temps de reconnaissance des plaques. Au niveau du parking intelligent, les pistes qui peuvent explorées sont: l’amélioration de la maquette en ajoutant un afficheur à l'entrée du parking qui montre un message de bienvenue et le nombre de places libres, l’utilisation d'un véhicule de test à pilotage automatique, le codage du programme de détection à lancer sur la carte Raspberry avec le langage C++ pour augmenter la performance de la barrière intelligente du parking.
