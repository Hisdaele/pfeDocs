\section{Présentation de l'entreprise}

Notre projet de fin d’études a été réalisé au sein d’une jeune et dynamique entreprise marocaine dont le siège se trouve dans la ville de Casablanca. Il s’agit de l’organisme \textbf{KF2Y Consulting}. Fondée en 2010, KF2Y met à la disposition de ses clients un ensemble de compétences et d’experts, pour le déploiement, la maîtrise et l’optimisation des systèmes d’information. L’entreprise propose aux autres entreprises de l'ensemble des secteurs économiques une approche nouvelle qui conjugue l'utilisation de méthodes novatrices et de bonnes pratiques, le recours à la technologie et l'expertise de son capital humain. Pour poursuivre son développement, KF2Y a misé sur 2 pôles:
    \begin{itemize}
        \item[•] \textbf{L’activité Consulting}: 
        Ce pôle est né du rapprochement de professionnels du conseil en management et systèmes d’information avec une expertise technologique qui consiste à aider ses clients, qu’ils soient entreprises privées, administrations publiques ou organisations non gouvernementales, à créer de la valeur via la construction et l’implémentation de solutions technologiques avec une ambition de construire des relations pérennes avec ses  clients-partenaires afin de mieux les connaître, mieux anticiper leurs besoins et mieux les servir. Cela se manifeste par la création de valeur ajoutée chez ses clients au niveau des ressources internes ou par une activité de sourcing qui consiste à intervenir dès la phase de recrutement de la ressource pour répondre à un besoin de mission chez nos clients.
        \item[•] \textbf{L’activité Recherche et Développement}: Elle vise à mobiliser des ressources internes dans une optique de création des solutions innovantes cherchant à répondre à un paramètre clé de cette mutation économique et digitale vu par le monde.
    \end{itemize}
 KF2Y Consulting couvre un champ d’applications très large et varié tel que :
    \begin{enumerate}
        \item[•] \textbf{Développement IT:} L'entreprise développe pour ses partenaires des applications mobiles et web en utilisant les technologies comme \textbf{Java EE, Angular, Angular JS, .Net, C\#, PL/SQL, SQL, ABAP/4,PHP,Python};
        \item[•] \textbf{ERP / Intégration de progiciels}
        \item[•] \textbf{Qualité logiciel et Testing}
        \item[•] \textbf{Big Data et Machine Learning}
        \item[•]  \textbf{Conseil et formation}
    \end{enumerate}
Dans chacun de ces domaines d'action, on retrouve des groupes de personnes qui travaillent sur des projets innovants. En ce qui concerne notre projet, nous avons collaboré avec les membres de l'équipe Big Data et Machine Learning.


La branche Big Data et Machine Learning est une nouvelle branche au sein de l’entreprise. Elle a été créée pour répondre aux besoins sans cesse croissants des systèmes intelligents. L’équipe attachée à cette branche est composée d’un manager qui suit l’avancé des projets, d’un responsable qui conduit techniquement les projets et de certains développeurs qui réalisent des applications intégrant des algorithmes de Big Data et de Machine Learning.