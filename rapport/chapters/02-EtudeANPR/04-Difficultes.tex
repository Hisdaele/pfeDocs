\section{Difficultés}
Depuis certaines années, les pays dans le monde normalisent le format de leurs plaques d’immatriculation. Cette normalisation intervient d’une part sur l’uniformisation des couleurs possibles et d’autre part la liste et le nombre de caractères que l’on peut trouver sur une plaque valide. Cette régularisation des plaques minéralogiques a rendu les systèmes ANPR plus spécifiques et donc plus performants. Néanmoins, comme tout système faisant du traitement d’image, les systèmes ANPR font face à des difficultés comme:
    \begin{itemize}
        \item[•] \textbf{Une mauvaise résolution de l’image} qui est la conséquence soit d’une caméra de mauvaise qualité soit de l’éloignement de la plaque par rapport à la caméra;
        \item[•] \textbf{Des images floues} principalement causées par le mouvement;
        \item[•] \textbf{Un mauvais éclairage et un faible contraste } à cause d'une surexposition, d'un reflet, ou d'ombres;
        Les objets qui cachent une partie de la plaque
        \item[•] \textbf{La différence de polices de caractères de la plaque} à cause de la fantaisie faite par certains usagers. \cite{wikianpr}   
    \end{itemize}
Si certains de ces problèmes peuvent être surmontés en agissant sur la partie logiciel du système, d’autres par contre nécessitent une intervention au niveau matériel. 
