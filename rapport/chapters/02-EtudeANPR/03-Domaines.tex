\section{Domaines d'application et Difficultés}
Depuis leur invention, les systèmes de reconnaissance automatique de plaques d’immatriculation n’ont cessé d’étendre leurs champs d’action. Ils sont même devenus incontournables dans les systèmes modernes et intelligents de gestion de trafic routier. Parmi les larges gammes d’applications de l’ANPR, on peut citer: 
    \begin{itemize}
        \item[•]\textbf{Aide à la reconnaissance des voitures volées}: le système est déployé sur le bord des routes, et réalise une comparaison en temps réel entre les voitures qui passent et la liste des voitures volées. Lorsqu’une correspondance est trouvée, une alerte est déclenchée pour informer l’agent de police de la voiture détectée et les raisons pour arrêter la voiture.
        \item[•]\textbf{Accès et paiement du stationnement dans les parkings} : le numéro de la plaque d’immatriculation est utilisé pour le payement du stationnement au parking pour les gens ayant des cartes près-payée pour les parkings, afin de calculer les frais de stationnement en comparant les temps d’entrée et de sortie au parking.
        \item[•]\textbf{Péage} : le numéro de la voiture est utilisé pour calculer les frais de voyage dans une route à péage, ou utilisé pour revérifier le billet.
        \item[•]\textbf{Contrôle d’accès} : l’ouverture automatique d’une porte pour les membres agrées dans une zone de sécurité. Ce genre de système est mis en place pour aider les agents de sécurité. Les événements sont enregistrés sur une base de données et peuvent être utilisés pour rechercher l’historique des événements en cas de besoin.
        \item[•]\textbf{Contrôle des frontières} : le numéro de la voiture est enregistré à l’entrée ou à la sortie du pays, et utilisé pour surveiller les passages frontaliers. Chaque véhicule est enregistré dans une base de données centrale et lié à des informations supplémentaires telle que les données relatives aux passeports. Il est utilisé pour suivre tous les passages frontaliers.
        \item[•]\textbf{Suivi du respect du code pénal de la route} : le numéro de plaque est utilisé pour produire une amende de violation de vitesse ou de feux rouges. Le processus manuel de préparation d’une amende de violation
        est remplacé par un processus automatisé qui réduit les surcharges et les délais. Les amendes peuvent être consultées et payées en ligne. \cite{HindeThesis}         
    \end{itemize} \par
Depuis certaines années, les pays dans le monde normalisent le format de leurs plaques d’immatriculation. Cette normalisation intervient d’une part sur l’uniformisation des couleurs possibles et d’autre part la liste et le nombre de caractères que l’on peut trouver sur une plaque valide. Cette régularisation des plaques minéralogiques a rendu les systèmes ANPR plus spécifiques et donc plus performants. Néanmoins, comme tout système faisant du traitement d’image, les systèmes ANPR font face à des difficultés comme:
    \begin{itemize}
        \item[•] \textbf{Une mauvaise résolution de l’image} qui est la conséquence soit d’une caméra de mauvaise qualité soit de l’éloignement de la plaque par rapport à la caméra;
        \item[•] \textbf{Des images floues} principalement causées par le mouvement;
        \item[•] \textbf{Les mauvaises conditions climatiques}: Les phénomènes naturels comme la pluie, la poussière et le brouillard peuvent empêcher une bonne capture des images et donc impacter négativement la précision d'un système ANPR.
        \item[•] \textbf{La différence de polices de caractères de la plaque} à cause de la fantaisie faite par certains usagers. \cite{wikianpr}   
    \end{itemize}
Si certains de ces problèmes peuvent être surmontés en agissant sur la partie logicielle du système, d’autres par contre nécessitent une intervention au niveau matériel. 